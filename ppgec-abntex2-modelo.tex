%% abtex2-modelo-trabalho-academico.tex, v-1.9.6 laurocesar
%% Copyright 2012-2016 by abnTeX2 group at http://www.abntex.net.br/ 
%%
%% This work may be distributed and/or modified under the
%% conditions of the LaTeX Project Public License, either version 1.3
%% of this license or (at your option) any later version.
%% The latest version of this license is in
%%   http://www.latex-project.org/lppl.txt
%% and version 1.3 or later is part of all distributions of LaTeX
%% version 2005/12/01 or later.
%%
%% This work has the LPPL maintenance status `maintained'.
%% 
%% The Current Maintainer of this work is the abnTeX2 team, led
%% by Lauro César Araujo. Further information are available on 
%% http://www.abntex.net.br/
%%
%% This work consists of the files abntex2-modelo-trabalho-academico.tex,
%% abntex2-modelo-include-comandos and abntex2-modelo-references.bib
%%

% ------------------------------------------------------------------------
% ------------------------------------------------------------------------
% abnTeX2: Modelo de Trabalho Academico (tese de doutorado, dissertacao de
% mestrado e trabalhos monograficos em geral) em conformidade com 
% ABNT NBR 14724:2011: Informacao e documentacao - Trabalhos academicos -
% Apresentacao
% ------------------------------------------------------------------------
% ------------------------------------------------------------------------

\documentclass[
	% -- opções da classe memoir --
	12pt,				% tamanho da fonte
	openright,			% capítulos começam em pág ímpar (insere página vazia caso preciso)
	oneside,			% para impressão em recto e verso. Oposto a oneside
	a4paper,			% tamanho do papel. 
	% -- opções da classe abntex2 --
	%chapter=TITLE,		% títulos de capítulos convertidos em letras maiúsculas
	%section=TITLE,		% títulos de seções convertidos em letras maiúsculas
	%subsection=TITLE,	% títulos de subseções convertidos em letras maiúsculas
	%subsubsection=TITLE,% títulos de subsubseções convertidos em letras maiúsculas
	% -- opções do pacote babel --
	english,			% idioma adicional para hifenização
	french,				% idioma adicional para hifenização
	spanish,			% idioma adicional para hifenização
	brazil				% o último idioma é o principal do documento
	]{ppgec-abntex2}

% ---
% Pacotes básicos 
% ---
\usepackage{times}			    % Usa a fonte Times New Roman
\usepackage[T1]{fontenc}		% Selecao de codigos de fonte.
\usepackage[utf8]{inputenc}		% Codificacao do documento (conversão automática dos acentos)
\usepackage{lastpage}			% Usado pela Ficha catalográfica
\usepackage{indentfirst}		% Indenta o primeiro parágrafo de cada seção.
\usepackage{color}				% Controle das cores
\usepackage{graphicx}			% Inclusão de gráficos
\usepackage{microtype} 			% para melhorias de justificação
% ---
		
% ---
% Pacotes adicionais, usados apenas no âmbito do Modelo Canônico do abnteX2
% ---
\usepackage{lipsum}				% para geração de dummy text
% ---

% ---
% Pacotes de citações
% ---
\usepackage[num]{abntex2cite}	% Citações padrão ABNT
\citebrackets[]
% hack para colocar colchetes nas referencias.
\makeatletter
\ifthenelse{\boolean{ABCIbiblabelonmargin}}
{
\renewcommand{\@biblabel}[1]%
{\ifthenelse{\equal{#1}{}}{}{{\citenumstyle #1\hspace{\biblabelsep}}}}
}
{
\renewcommand{\@biblabel}[1]%
{%
\ifthenelse{\equal{#1}{}}
{}
{%
\def\biblabeltext{{\citenumstyle [#1]\hspace{\biblabelsep}}}%
\settowidth{\ABCIauxlen}{\biblabeltext}%
\ifthenelse{\lengthtest{\ABCIauxlen<\minimumbiblabelwidth}}
{\setlength{\ABCIauxlen}{\minimumbiblabelwidth-\ABCIauxlen}}
{\setlength{\ABCIauxlen}{0cm}}%
{\biblabeltext\hspace{\ABCIauxlen}}% 
}%
}%
}
\makeatother
% ---
% Informações de dados para CAPA e FOLHA DE ROSTO
% ---
% ---
% Informações de dados para CAPA e FOLHA DE ROSTO
% ---

\titulo{Modelo Canônico de\\ Trabalho Acadêmico com \abnTeX customizado para o PPGEC}
\autor{Victor Kléber Santos Leite Melo}
\local{Recife}
\data{setembro de 2016}
\orientador{Prof. Dr. Lauro César Araujo}
\coorientador{Equipe \abnTeX}
\instituicao{
  \SingleSpacing
  Universidade de Pernambuco \\
  Escola Politécnica de Pernambuco \\
  Programa de Pós-Graduação Acadêmica em Engenharia de Computação
}
\tipotrabalho{Dissertação de Mestrado}
% O preambulo deve conter o tipo do trabalho, o objetivo,
% o nome da instituição e a área de concentração
\preambulo{Dissertação apresentada ao Programa de Pós-Graduação acadêmico em ENGENHARIA DE COMPUTAÇÃO da Universidade de Pernambuco como requisito parcial para obtenção do título de Mestre em Engenharia de Computação.}
% ---



% ---
% Configurações de aparência do PDF final

% alterando o aspecto da cor azul
\definecolor{blue}{RGB}{41,5,195}

% informações do PDF
\makeatletter
\hypersetup{
     	%pagebackref=true,
		pdftitle={\@title}, 
		pdfauthor={\@author},
    	pdfsubject={\imprimirpreambulo},
	    pdfcreator={LaTeX with abnTeX2},
		pdfkeywords={abnt}{latex}{abntex}{abntex2}{trabalho acadêmico}, 
		colorlinks=true,       		% false: boxed links; true: colored links
    	linkcolor=blue,          	% color of internal links
    	citecolor=blue,        		% color of links to bibliography
    	filecolor=magenta,      		% color of file links
		urlcolor=blue,
		bookmarksdepth=4
}
\makeatother
% --- 

% --- 
% Espaçamentos entre linhas e parágrafos 
% --- 

% O tamanho do parágrafo é dado por:
\setlength{\parindent}{1.3cm}

% Controle do espaçamento entre um parágrafo e outro:
\setlength{\parskip}{0.2cm}  % tente também \onelineskip

% ---
% compila o indice
% ---
\makeindex
% ---

%%% -----
%%% Formato de cabeçalho/rodapé romano nos elementos pré-textuais
%%% -----

%% Novo estilo
\makepagestyle{estilo_pretextual} %%% escolha um nome
\makeevenhead{estilo_pretextual}{}{}{\ABNTEXfontereduzida \textit \thepage}
\makeoddhead{estilo_pretextual}{}{}{\ABNTEXfontereduzida \textit \thepage}

%% Customiza comando \pretextual
\renewcommand{\pretextual}{

  \aliaspagestyle{chapter}{estilo_pretextual}% customizing chapter pagestyle
  \pagestyle{estilo_pretextual}
  \aliaspagestyle{cleared}{empty}
  \aliaspagestyle{part}{estilo_pretextual}
}

% ---
% Ajusta a marca \textual para que a numeração volte a ser arábica
% nos elementos textuais
\let\oldtextual\textual        % copia o comando \textual anterior para \oldtextual
\renewcommand{\textual}{%
  \oldtextual%
  \pagenumbering{arabic} % volta à numeração arábica
}
% ---



% ----
% Início do documento
% ----
\begin{document}

% Seleciona o idioma do documento (conforme pacotes do babel)
%\selectlanguage{english}
\selectlanguage{brazil}

% Retira espaço extra obsoleto entre as frases.
\frenchspacing 

% ----------------------------------------------------------
% ELEMENTOS PRÉ-TEXTUAIS
% ----------------------------------------------------------


% ---
% Capa
% ---
\imprimircapa
% ---

% ---
% Folha de rosto
% (o * indica que haverá a ficha bibliográfica)
% ---
\thispagestyle{empty} 
\imprimirfolhaderosto*
% ---

% Inicia a numeração dos elementos pre textuais após a folha de rosto
\pagenumbering{roman} %%% ou \pagenumbering{Roman}

% ---
% Inserir a ficha bibliografica
% ---

% Porém, a biblioteca da sua lhe fornecerá um PDF
% com a ficha catalográfica definitiva após a defesa do trabalho. Quando estiver
% com o documento, salve-o como PDF no diretório do seu projeto e substitua todo
% o conteúdo de implementação deste arquivo pelo comando abaixo:
%
% \begin{fichacatalografica}
%     \includepdf{fig_ficha_catalografica.pdf}
% \end{fichacatalografica}

% ---

% ---
% Inserir errata
% ---
\include{pretextuais/errata}

% ---

% ---
% Inserir folha de aprovação
% ---

% Isto é um exemplo de Folha de aprovação, elemento obrigatório da NBR
% 14724/2011 (seção 4.2.1.3). Você pode utilizar este modelo até a aprovação
% do trabalho. Após isso, substitua todo o conteúdo deste arquivo por uma
% imagem da página assinada pela banca com o comando abaixo:
%
% \includepdf{folhadeaprovacao_final.pdf}
%

% ---

% ---
% Dedicatória
% ---

\begin{dedicatoria}
   \vspace*{\fill}
   \centering
   \noindent
   \textit{Parte do trabalho onde o autor pode dedicar a sua obra. Não é obrigatório fazer uma dedicatória, mas muitos autores sentem a necessidade de homenagear alguém. } \vspace*{\fill}
\end{dedicatoria}

% ---

% ---
% Epígrafe
% ---
\begin{epigrafe}
    \vspace*{\fill}
	\begin{flushright}
		\lipsum[1-1]
	\end{flushright}
\end{epigrafe}
% ---

% ---
% RESUMOS
% ---

\include{pretextuais/resumos}


% ---
% inserir o sumario
% ---
\pdfbookmark[0]{\contentsname}{toc}
\tableofcontents*
\cleardoublepage
% ---


% ---
% inserir lista de ilustrações
% ---
\pdfbookmark[0]{\listfigurename}{lof}
\listoffigures
\cleardoublepage
% ---

% ---
% inserir lista de tabelas
% ---
\pdfbookmark[0]{\listtablename}{lot}
\listoftables
\cleardoublepage
% ---

% ---
% inserir lista de siglas e simbolos
% ---

\begin{siglas}
	\addcontentsline{toc}{chapter}{\listadesiglasname}
	
  \item[ABNT] Associação Brasileira de Normas Técnicas
  \item[abnTeX] ABsurdas Normas para TeX
  \item[$ \Gamma $] Letra grega Gama
  \item[$ \Lambda $] Lambda
  \item[$ \zeta $] Letra grega minúscula zeta
  \item[$ \in $] Pertence
\end{siglas}


% ---



% ---
% Agradecimentos
% ---
\include{pretextuais/agradecimentos}
% ---


% ----------------------------------------------------------
% ELEMENTOS TEXTUAIS
% ----------------------------------------------------------
\textual
\pagestyle{simple}
\aliaspagestyle{chapter}{simple}


\include{conteudo/introducao}
\chapter{Fundamentação teórica}

\lipsum[1-2]

% FALTA AJUSTAR \subsubsubsection{Exemplo subsubsubsection}
\include{conteudo/desenvolvimento}
\chapter{Resultados e discussão}

\lipsum[1-1]


\chapter{Conclusão}

\lipsum[1-1]


% ----------------------------------------------------------
% ELEMENTOS PÓS-TEXTUAIS
% ----------------------------------------------------------
\postextual
% ----------------------------------------------------------

% ----------------------------------------------------------
% Referências bibliográficas
% ----------------------------------------------------------
\bibliography{referencias}

% ----------------------------------------------------------
% Glossário
% ----------------------------------------------------------
%
% Consulte o manual da classe abntex2 para orientações sobre o glossário.
%
%\glossary

% ----------------------------------------------------------
% Apêndices
% ----------------------------------------------------------

% ---
% Inicia os apêndices
% ---
\begin{apendicesenv}

% ----------------------------------------------------------
\chapter{Quisque libero justo}
% ----------------------------------------------------------

\lipsum[50]

% ----------------------------------------------------------
\chapter{Nullam elementum urna vel imperdiet sodales elit ipsum pharetra ligula
ac pretium ante justo a nulla curabitur tristique arcu eu metus}
% ----------------------------------------------------------
\lipsum[55-57]

\end{apendicesenv}
% ---


% ----------------------------------------------------------
% Anexos
% ----------------------------------------------------------

% ---
% Inicia os anexos
% ---
\begin{anexosenv}


% ---
\chapter{Morbi ultrices rutrum lorem.}
% ---
\lipsum[30]

% ---
\chapter{Cras non urna sed feugiat cum sociis natoque penatibus et magnis dis
parturient montes nascetur ridiculus mus}
% ---

\lipsum[31]

% ---
\chapter{Fusce facilisis lacinia dui}
% ---

\lipsum[32]

\end{anexosenv}

%---------------------------------------------------------------------
% INDICE REMISSIVO
%---------------------------------------------------------------------
\phantompart
\printindex
%---------------------------------------------------------------------

\end{document}
